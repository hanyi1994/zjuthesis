

\chapter{总结与展望}

可视化交通仿真作为计算机动画、虚拟现实技术和交通仿真有机融合的一个重要研究领域,近年来得益于自动驾驶行业的快速发展,已广泛应用于交通流中智能体仿真模拟和高保真视觉呈现,而本文系列工作的灵感来源正是作者在自动驾驶公司实习时业界提出的需求。因此,本文主要从自动驾驶中的World-Sim的需求出发,以提高仿真结果多样性和定制化结果生成效率、保证运动真实平滑和用户交互体验为目标,深入研究了交互式交通仿真与轨迹编辑框架、基于时-空关键帧控制的交通仿真、基于用户引导注入非跟驰行为的交通仿真来提升个体的行为多样性,又研究了基于简化社会里模型的混合交通仿真来提升个体的种类多样性。接下来我们对本文的工作进行总结,并展望未来工作。

%基于简化社会力模型的多智种类能体混合交通仿真问题,多样性和非常规性注入的实时交互式轨迹编辑问题,利用由粗到细优化策略求解时-空关键帧控制的交通仿真问题,以及通过交互引入到车行为的非跟驰交通仿真问题。接下来我们将对本文的工作进行总结,并展望未来工作。

%群组动画,作为计算机动画中的一个重要研究领域,近年来得益于自动驾驶行业的快速发展,已广泛应用于交通流中智能体的仿真模拟。本文由交互式的交通仿真与轨迹编辑技术展开,深入研究了基于简化社会力模型的多智种类能体混合交通仿真问题,多样性和非常规性注入的实时交互式轨迹编辑问题,利用由粗到细优化策略求解时-空关键帧控制的交通仿真问题,以及通过交互引入到车行为的非跟驰交通仿真问题。接下来我们将对本文的工作进行总结,并展望未来工作。


\section{本文工作总结}

为了高效、直观地生成包含非常规行为的交通轨迹数据,我们提出了一个实时交互式交通仿真和轨迹编辑方法,命名为TraEDITS。TraEDITS中集成了一个基于优化的交通仿真模块和一个全局路径规划模块。在全局路径规划模块,TraEDITS将静态场景离散化为网格,使用胶囊状近似车道几何存储到网格中,并将用户指定的一系列关键位置映射到对应的网格单元内,使用启发式路径规划算法生成用自定义的参考路径;在基于优化的交通仿真模块,车辆的状态和运动更新使用非欧的坐标表示,使其从静态车道的限制中解耦出来,并在能量优化的过程中暴露出若干可供修改的数值接口供用户实时修改。TraEDITS大大减少了用户获得预期交通轨迹数据的时间,且能生成过往方法或现有数据集中鲜少见到的行为。值得一提的是,TraEDITS框架已应用在嬴彻科技(Inceptio)公司内部的World-Sim流水线中。

为了为用户提供更多维度的轨迹编辑控制,我们提出了一种允许用户在仿真进行过程中使用时-空关键帧控制车辆行为和轨迹编辑的方法。我们提出了一个新颖的基于社会力的交通仿真框架,在非欧坐标系和欧式坐标系中同时表示和更新车辆的状态和运动。用户在仿真过程中通过交互为特定个体设定关键帧以约束其行为。基于给定的关键帧,我们使用了一个由粗到细的优化过程求解关键帧的约束。在沿着参考路径方向的离散化状态-时间空间中构建有向图,将关键帧映射到有向图的节点中,通过启发式搜索规划出一条粗粒度的轨迹;从粗粒度的轨迹中提取出信息来初始化伴随法,得益于社会力模型的可微性进行梯度下降,保证梯度快速且稳定地收敛,进一步得到更精细、平滑的轨迹。本方法使轨迹编辑过程更灵活、直观,满足了用户在编辑过程中的时-空二维度的控制要求,同时也保证了生成轨迹的质量。

为了突破传统的跟驰模型将倒车行为引入交通仿真,我们提出了一种利用用户指定的关键状态控制车辆倒车运动的交互式仿真框架,生成同时包含跟驰行为和非跟驰行为的车辆轨迹数据和交通场景。在离散化的场景表示中,基于用户拖拽生成的关键状态信息,我们使用考虑车辆运动学的混合A*算法生成具有前向和反向导航的定制化参考路径。在考虑了车辆运动学、自驱动、路径保持和碰撞避免的情况下,利用速度空间采样和能量最优化更新车辆状态。我们特殊设计了碰撞避免能量项,使其同时包含考虑安全跟车距离的软约束和考虑车辆几何信息的边界硬约束;而为了进一步明确各类车辆在运动过程中的交互优先级,我们又定义了特殊的车辆交互规则。本方法可以在发生非跟驰运动的交通场景中,保证倒车运动的车辆和其邻车对其的反馈更平滑、真实。

为了避免大量“专人专用”的、无实际物理意义的模型参数,我们提出了一种基于简化社会力模型的混合交通仿真方法,旨在引入一个更简洁、更易扩展的框架来统一仿真不同种类智能体的行为和相互影响。我们基于面向对象的思想设计了一个用于表达多种类智能体之间行为共性的基类,并通过分析先前方法所用的社会力模型的趋势设计了一个简化的社会力模型用于计算个体交互行为中的各类吸引力和排斥力,最终基类到不同种类智能体的派生被建模为使用个体属性参数化社会力模型系数的过程。本工作在保证混合交通参与者的多态性的同时,也大大提高了用户调参的效率和模型的计算性能。


\section{未来工作展望}

在未来的工作中,我们将继续围绕着交通仿真与实时交互式编辑技术展开,在提高用户编辑效率、保证交互过程简洁的前提下,更进一步生成足够真实的、多样且非常规的交通轨迹数据与场景。可行的研究方向主要包括以下几个方面:

首先,利用采集的交通数据集来生成更真实的车辆运动。我们的混合交通仿真方法和交互式交通仿真方法均基于经验建模,只能在视觉上保证结果的相对真实。目前的采集交通数据集包含多源数据,例如轨迹数据、摄像头数据等等,其中蕴含着很多可供学习的驾驶员行为或是多智能体交互影响等待我们去挖掘。但由于采集的交通数据存在在有车道线的道路上行为单一,而在无车道线的非结构化地区又难以对车辆运动进行标注的情况。因此如何去获取、标注并使用更丰富的多源交通数据,去指导我们的交互式交通仿真方法生成更真实的非常规车辆行为(例如U 型掉头、K型掉头、急转弯、借道、倒车等等),是十分值得尝试的。

其次,针对大规模车流进行快速方便的实时编辑,而不增加用户的交互负担。当前我们的交互式交通仿真方法均以个体为单位施加约束以改变其运动轨迹,而邻车是根据被编辑的车辆自适应调整其运动。但当用户要编辑的车辆数量增多时,一个一个去编辑会导致用户同样需要花费很多的人力和时间成本,违背了我们引入交互式编辑以直观、高效控制车辆运动的初衷。在人群动画中,有工作~\cite{kim2014interactive, zhang2020crowd}使用基于cage或网格的变形来快速调整人群的轨迹以适配场景,然后对调整后的人群进行后续的碰撞避免优化。若要引入类似的批操作方法用于控制车辆的运动,则还要额外考虑车辆的运动学模型、车道信息、交通规则等约束。这与人群仿真中个体运动具有高自由度的特点相异,如何求解这些约束也是一个值得今后探索的方向。

再次,如何仿真道路中行驶车辆之间的交互和博弈,也是一个具有挑战性方向。在我们使用关键帧控制的交通仿真工作中,关键帧的约束可能会由于拥挤的原因而无法满足;在引入到车的交互式交通仿真工作中,跟驰车辆只能够在非跟驰运动的车辆驶离视线或结束运动后才继续运动,即使其有足够的时间先行通过。尽管这些问题能够通过迭代式地用户编辑解决,而在现实生活中驾驶员会通过车辆灯光、手势等信息来传递信息:如果车辆因前车过慢而达不到预期速度,会提前开启转向灯鸣笛变道超车;而当车辆在面对倒车等容易引起堵塞的行为时,也会互相打手势告知让对方先行,或是鸣笛提醒对方我方将要先行。将驾驶员之间的交互、博弈考虑到交通仿真中,会具有更广泛的实际应用价值和理论研究意义。

最后,虽然本文在所有的方法中都将“碰撞避免”视为约束个体运动的重要影响因素,但是使用World-Sim来生成交通事故场景,能够大大降低原先使用Log-Sim来进行自动驾驶算法鲁棒性测试的数据采集成本,极大程度上保护公共财产和人身安全。但也正如先前在章节\ref{section:intro_challenge}和章节\ref{chapter:traedits}开头提及的,使用可视化交通仿真来产生视觉上可靠的、不同破坏程度的事故数据,不仅需要强大的群体行为仿真技术,更需要高效的碰撞检测、特效画面渲染、ragdoll模拟、分布式计算等技术的协同呈现。因此,融合不同方向的技术来生成更丰富、视觉保真的交通场景依然任重道远。
